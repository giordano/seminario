\documentclass[12pt,a4paper,fleqn,draft]{article}
\usepackage[T1]{fontenc}
\usepackage[utf8]{inputenc}
\usepackage[italian]{babel}

\usepackage{amsmath,amssymb,siunitx,layaureo,graphicx,hyperref}

\sisetup{per-mode=symbol,
  inter-unit-separator={}\cdot{},
  exponent-product=\cdot,
  output-product=\cdot,
  separate-uncertainty=true
}
\DeclareSIUnit\year{yr}

\newcommand{\figura}[1]{
  \begin{figure}[h!]
    \includegraphics[height=12pt]{Immagini/#1}
  \end{figure}
}

\title{L'energia oscura\\(traccia della presentazione)}
\author{Mosè Giordano}
\date{16 settembre 2011}

\begin{document}
\maketitle
L'energia oscura è un'ipotetica forma di energia che riempie l'Universo. Nel
seguente grafico a torta vediamo una stima della concentrazione di energia
oscura, materia oscura e materia ordinaria nell'Universo.
\figura{DarkMatterPie}

È stato necessario, alla fine degli scorsi anni '90, introdurre l'energia oscura
all'interno della teoria cosmologica standard perché sono state raccolte delle
evidenze sperimentali che mostrano che l'Universo si sta espandendo in maniera
accelerata. Fino ad allora si sapeva che l'Universo si sta espandendo, ma si
riteneva che lo facesse in modo costante. Vediamo in questa rappresentazione
artistica, prodotta dalla NASA, che intorno a circa $5$ miliardi di anni fa
l'espansione avrebbe iniziato a subire un'accelerazione.
\figura{800px-CMB_Timeline300_no_WMAP}

Alcune supernove hanno una forte riga di emissione di idrogeno nel proprio
spettro, mentre per altre supernove questa riga è assente. Le prime sono dette
di tipo II, le altre di tipo Ia. Si è quindi scoperto che le supernove di tipo
II derivano dall'esplosione di una stella molto massiccia nello stadio finale
della propria vita, quando ancora è presente una grossa nube di idrogeno nella
loro atmosfera. Le supernove di tipo I si dividono a loro volta in Ia, Ib e
Ic. Lo spettro delle Ia presenta, nello spettro di assorbimento, una forte riga
di silicio ionizzato, che manca per le Ib e Ic. [Le Ib hanno una forte riga di
assorbimento dell'elio, le Ic no. Secondo le teorie attuali, le Ib e Ic derivano
dall'esplosione di una stella massiccia, come le II, ma a cui sono stati portati
via (vento stellare o stella compagna) gli strati esterni prima di
esplodere. Alle Ic è stata sottratta molta massa, quindi non ci sono più neanche
gli strati di elio, per cui questo elemento è assente dallo spettro di
assorbimento.] A differenza degli altri tipi di supernove, le Ia non sono state
individuate in regioni in cui si stanno formando nuove stelle, quindi
probabilmente non costituiscono la fase conclusiva della vita di una massiccia
stella supergigante. Piuttosto si ritiene che le supernove Ia derivino
dall'esplosione termonucleare di una nana bianca che forma un sistema binario,
per esempio con una gigante rossa. Se la nana bianca sottrae massa alla sua
compagna può a un certo punto superare il limite di Chandrasekhar e riattivare
le reazioni nucleari che portano, in seguito, all'esplosione della
stella. [Prima di esplodere la nana bianca era composta principalmente da
carboni e ossigeno. Dalla reazione di fusione nucleare del carbonio deriva il
silicio che caratterizza lo spettro delle supernove Ia. In queste stelle sono
quasi completamente assente, invece, l'idrogeno e l'elio, per questo motivo non
compaiono negli spettri.] Poiché tutte le supernove Ia hanno pressoché la stessa
massa e stessa composizione chimica, si ritiene che esplodano tutte allo stesso
modo. In particolare, avrebbero tutte la stessa curva di luce (andamento della
luminosità assoluta in funzione del tempo trascorso dall'esplosione), con un
picco intorno a -19.9 magnitudini assolute. Poiché esiste una relazione
[$m-M=5\log d - 5$] che lega magnitudine assoluta [$M$, in questo caso nota],
magnitudine apparente [$m$, misurabile nelle osservazioni] e distanza [$d$,
incognita], misurando la magnitudine apparente del picco della curva di luce di
una queste supernove è possibile stimare la distanza da noi di questi corpi.

Due gruppi di ricerca su supernove Ia [Riess 1998, Perlmutter 1999] scoprirono,
misurando la distanza con il metodo sopra esposto, che alcune supernove molto
lontane apparivano più distanti di quanto si prevedesse da misure basate sul
redshift cosmologico. Adattando i dati raccolti a un modello cosmologico
scoprirono che per giustificare questa apparente discrepanza era necessario
ammettere l'esistenza di una certa forma di energia che comporta
un'accelerazione nell'espansione dell'Universo.
\figura{fitting}

Da allora sono state raccolte molte altre evidenze dell'esistenza dell'energia
oscura con diverse tecniche, delle quali non avremo il tempo di parlare:
\begin{itemize}
\item età dell'Universo [si stima che le stelle più vecchie ancora in vita
  abbiano un'età maggiore di $t_s = \SI{11}{\giga\year}$. Considerando un
  Universo dominato solo dalla materia comune si può calcolare che questo
  avrebbe un'età di circa $t_0 = \SI{9}{\giga\year}$, in contrasto con l'età
  delle stelle più vecchie. Ammettendo l'esistenza dell'energia oscura si ricava
  $t_0 = \SI{13.7}{\giga\year}$, valore compatibile con le osservazioni];
\item radiazione cosmica di fondo;
\item struttura a grande scala dell'Universo;
\item lensing gravitazionale;
\item oscillazioni acustiche barioniche;
\item conteggio degli ammassi di galassie.
\end{itemize}

Su cosa sia realmente l'energia oscura al momento ci sono solo speculazioni. Le
due ipotesi più diffuse sono quelle della costante cosmologica e della
quintessenza. Esistono anche altre ipotesi alternative che spesso finiscono con
il coincidere con la quintessenza oppure non sono compatibili con le
osservazioni.

La cosmologia si poggia sulla teoria della Relatività Generale. Qui ricaveremo
le equazioni che descrivono l'energia oscura usando la meccanica classica
newtoniana. Non è un metodo rigoroso, però i risultati a cui arriveremo sono gli
stessi predetti in maniera più formale dalla Relatività Generale. [Peraltro con
la riformulazione di Élie Cartan della meccanica newtoniana si può far vedere
che la cosmologia di Newtonia è rigorosa quanto (e più generale di) quella di
Friedmann. Vedi Frank J. Tipler, ``Rigorous Newtonian cosmology'', in American
Journal of Physics -- October 1996 -- Volume 64, Issue 10, pp. 1311-1315. DOI:
\url{http://dx.doi.org/10.1119/1.18398}] In particolare l'energia oscura deriva
da un modello che presuppone il principio cosmologico: l'Universo è omogeneo
(uguale in ogni suo punto) e isotropo (uguale in ogni direzione). Chiaramente
questo non è valido sempre, ma sulle grandi scale astronomiche (decine o
centinaia di Mpc e oltre) questa ipotesi vale con buona approssimazione.

Consideriamo un mezzo sferico, omogeneo e isotropo in espansione che ha densità
$\rho$, raggio $r$ e massa $M = 4\pi\rho r^3/3$. Consideriamo inoltre una
particella di massa $m$ sulla sua superficie. Il modulo della forza
gravitazionale che la particella risente per azione del mezzo è $F = \dots$,
l'energia potenziale è $V = \dots$, l'energia cinetica $T = \dots$, quindi
l'energia meccanica totale è $U = \dots$.  Poiché il mezzo si sta espandendo
effettuiamo un cambio di coordinate: poniamo $\vec{r}(t) = R(t) \vec{x}$, in cui
$\vec{x}$ è la coordinata di comoto (?), costante, e $R(t)$ è il fattore di
scala che misura il tasso di espansione del mezzo. [L'ipotesi di omogeneità
assicura che $R$ dipenda solo dal tempo.]  Riarrangiando i termini
dell'equazione dell'energia e ponendo $kc^2 = \dots$ otteniamo l'equazione di
Friedmann $H^2 = \dots$. Il parametro $k$ non deve dipendere da $x$ perché
nell'equazione nessun termine dipende da $x$ e deve essere preservata
l'omogeneità, $k$ non dipende neanche dal tempo. L'omogeneità inoltre comporta
che $U$ è costante per una fissata particella ma cambia, al variare di $x$, come
$x^2$. L'equazione di Friedmann non è covariante, non vale per qualsiasi cosa
che si trovi nell'Universo, perché si basa sulla validità del principio
cosmologico, quindi può essere applicato sulle grandi scale. Pertanto,
l'espansione dell'Universo non significa che noi adesso ci stiamo espandendo
perché l'espansione ha luogo solo se il moto degli oggetti è governato
dall'effetto gravitazionale cumulativo di una distribuzione omogenea di materia
fa di loro. Per esempio, gli atomi del nostro corpo sono soggette alle forze
nucleari, la gravità è del tutto irrilevante per essi. Invece lo spazio che
separa le galassie si sta espando portando esse con sé.

Dalla prima legge della termodinamica [$dE + pdV = TdS$, si assume un'espansione
reversibile a entropia costante $dS = 0$] si ricava l'equazione del fluido
$\dot{\rho} + \dots$. Per determinare l'andamento di $\rho$ rispetto al tempo
abbiamo bisogno di conoscere la pressione. Per questo motivo introdurremo una
funzione del tipo $p \equiv p(\rho)$ che lega le due quantità.

Derivando l'equazione di Friedmann rispetto al tempo e inserendo l'equazione del
fluido si ricava anche l'equazione di accelerazione $\ddot{R}/R = \dots$. La
materia ordinaria non esercita pressione e ha densità positiva, quindi
$\ddot{R}<0$ e non si riesce a spiegare l'espansione dell'Universo. Lo stesso
vale se $p>0$. Queste equazioni non sono tutte fra loro indipendenti, avendone
due si può sempre ricavare la terza.

Con il metodo da noi sviluppato non è evidente, ma con la teoria della
Relatività Generale si può far vedere che il parametro $k$ che abbiamo
introdotto descrive la curvatura dell'Universo. Senza energia oscura:
\begin{itemize}
\item Universo chiuso: collasserà su se stesso (Big Crunch));
\item Universo piatto: continuerà a espandesi indefinitamente;
\item Universo aperto: esploderà (Big Rip).
\end{itemize}

Per giustificare l'accelerazione nell'espansione dell'Universo modifichiamo la
forza di gravità aggiungendo un termine repulsivo del tipo legge di Hooke:
$\Lambda c^2\vec{r}/3$. [Il potenziale di questo termine va come $x^2$, come
richiesto dall'ipotesi di omogeneità.] Ripetendo gli stessi calcoli di prima si
ottiene l'equazione di Friedmann modificata $H^2 = \dots$. Il termine aggiuntivo
che ne risulta è la costante cosmologica che, in questo modello, descrive il
comportamento dell'energia oscura. Possiamo riscrivere l'equazione nella stessa
forma di prima includendo il termine della costante cosmologica in quello della
pressione. La costante cosmologica $\Lambda$ fu introdotta già da Einstein nel
1917 per provare a spiegare un Universo statico. Dovette però ritirarla quando
si scoprì che l'Universo non era statico ma è in espansione, ammettendo di aver
commesso il più grosso errore della sua vita.

Possiamo immaginare l'Universo come riempito da un fluido. [Non è un'ipotesi
solo semplificatrice: il moto delle stelle (?) è descritto dalla meccanica
statistica, così come succede per il moto dei fluidi.] Questo fluido è composto
da (almeno) tre componente e ha densità $\rho = \dots$ , con $\rho_m = \dots$
(barioni (impropriamente): tutto ciò che fa parte del modello standard delle
particelle (protoni + neutroni + elettroni); materia oscura: materia che non
emette o diffonde radiazione elettromagnetica. La materia è tutto ciò che
tipicamente si muove non relativisticamente, l'energia cinetica è minore
dell'energia a riposo, quindi non ha pressione (polvere)), $\rho_r = \dots$
(radiazione: cioè che si muove alla velocità della luce o quasi, quindi
particelle prive di massa (fotoni) o quasi (neutrini, questi inoltre
interagiscono scarsamente con la materia)) e $\rho_\Lambda = \dots$ (energia
oscura). Notiamo che la densità di energia oscura è, in questo modello, costante
se $\Lambda$ è costante.

Come abbiamo già detto, per risolvere $\rho$ in funzione del tempo è necessario
introdurre una relazione fra pressione e densità. Questa è data dall'equazione
di stato che possiamo scrivere genericamente come $p_i = \dots$.  Inserendo
questa relazione nell'equazione dei fluidi si trova che per ciascuna componente
abbiamo $\rho_i = \dots$.

Per la materia abbiamo già detto che $w = 0$, quindi $\rho \propto \dots$ come è
normale attendersi poiché si la quantità di materia rimane costante nel
tempo. Per la radiazione si può dimostrare che si ha $w = 1/3$, quindi
$\rho \propto \dots$. Infine, ponendo $w=-1$ per l'energia oscura abbiamo
$\rho \text{ costante}$. La figura mostra, allo stato delle conoscenze attuali,
l'andamento delle densità delle tre componenti nel corso del tempo a partire dal
Big Bang. Inizialmente l'Universo era dominato dalla radiazione, ma dopo
relativamente poco tempo la materia ha avuto il sopravvento (perché
$\rho_m \propto 1/R^3$, $\rho_r \propto 1/R^4$), ma da circa 5 miliardi di anni
la densità di materia è inferiore a quella dell'energia oscura e quindi è
iniziata l'accelerazione nell'espansione dell'Universo.
\figura{evoluzione_densita}

Si possono introdurre dei parametri adimensionali che rendono l'idea di quale
sia la concentrazione delle varie componenti del fluido cosmologico. Se
nell'equazione di Friedmann poniamo $k=0$ otteniamo la densità critica, ovvero
la densità che deve possedere complessivamente il fluido cosmologico affinché la
curvatura dell'Universo sia piatta. Allora per ogni componente del fluido
definiamo il parametro di densità $\Omega_i = \dots$. Possiamo anche definire un
parametro di densità associato alla curvatura $\Omega_k = \dots$, direttamente
legato alla curvatura ($\Omega_k < 0 \implies k >0$, $\Omega_k = 0 \implies
k=0$, $\Omega_k > 0 \implies k<0$) e che permette di riscrivere l'equazione di
Friedmann in questo modo $\Omega + \Omega_k \equiv \dots$.

Questi parametri di densità sono le grandezze che entrano nei modelli in base ai
quali si effettuano i fit dei dati osservazionali e dei quali, dunque, si
possono ottenere delle stime sperimentali. Questi che vediamo nella figura sono
i limiti imposti dai risultati di esperimenti basati su alcune delle tecniche
elencate precedentemente e vediamo che $\Omega_m \approx \num{0.3}$
($\Omega_{\textup{b},0} \approx \num{0.05}$,
$\Omega_{\textup{dm},0} \approx \num{0.25}$), $\Omega_r \approx \num{5e-5}$
($\Omega_\nu \approx 0$) e $\Omega_\Lambda \approx \num{0.7}$. Quindi l'energia
oscura domina l'Universo, come avevamo visto nel grafico a torna
iniziale. Inoltre l'analisi della radiazione cosmica di fondo fornisce forti
evidenze del fatto che $k=0$, cioè che la curvatura dell'Universo sia piatta.
\figura{confcmbclust}

La costante cosmologica ha densità costante in spazio e tempo. Se invece
ammettiamo che possa variare poniamo
$p_{\textup{Q}} = w_{\textup{Q}}\rho_{\textup{Q}}c^2$ con
$w_{\textup{Q}} \neq 1$ e $w_{\textup{Q}} \lesssim -1/3$ affinché si abbia
l'espansione accelerata (cioè $\ddot{R}>0$). [Per soddisfare questa condizione
dobbiamo avere $p < -\rho c^2/3$, $\rho_\Lambda$ è solo una delle componenti di
$\rho$, sebbene la più importante, per questo ho usato il simbolo $\lesssim$
invece di $<$.] In questo caso l'energia oscura si chiama quintessenza. Allo
stato delle conoscenze attuali non si sa se l'energia oscura si comporti come
una costante cosmologica oppure la sua densità sia cambiata, seppure
leggermente, nel corso del tempo.

L'esistenza dell'energia oscura ha importanti conseguenze sul destino
dell'Universo. Come abbiamo visto, adesso starebbe guidando un'accelerazione
nell'espansione. Quale sarà il destino ultimo dell'Universo dipende fortemente
dalle reali densità dei vari componenti del fluido cosmologico e dalla curvatura
dell'Universo.
\figura{500px-Friedmann_universes}

\end{document}

%%% Local Variables:
%%% mode: latex
%%% TeX-master: t
%%% End:
